In this thesis, I will study results in the field of harmonic and Fourier analysis. I will be concentrating on results that prove that Poisson integrals in the case of $L^p$ spaces and Poisson-Stieltjes integrals in the case of finite Borel measures are a way to extend a function describing some boundary value at the boundary of a half space $\RR^n\times\{0\}$, to the whole of the half space in a way that it satisfies the Laplace equation $\Delta u=\sum_i \frac{\partial^2 u}{\partial x_i^2}=0$, while converging towards the function at the boundary. This work will follow mostly the results given in the \textit{Introduction to Fourier Analysis on Euclidean Spaces} by E. M. Stein and G. Weiss \cite{stein_weiss}.

The problem that is investigated in this thesis is a special case of the Dirichlet problem. The problem poses a question: can we find a solution for a partial differential equation (PDE), such that it will converge to some specified function at the boundary of the space? For the purpose of this thesis, the PDE that will be considered is the Laplace equation, and the space will be the half space $\RR^n\times\RR_+$.

This question relates to many areas of mathematical analysis and is deeply connected to mathematical physics. In mathematics, these types of results are frequently used in the analysis of PDEs to extend certain weaker solution concepts to boundary value problems. This allows embedding less regular initial value data, which may not satisfy the boundary condition in the classical sense of functions. These are connected to physical problems through potential theory, where, for example, gravitational and electrostatic potentials can be described by harmonic functions.

In the rest of this thesis, I will first introduce some of the essential tools that are used in the construction of the Poisson kernel in Chapter \ref{ch:2}. This will also include showing some properties of the Poisson kernel that will be needed later. Chapter \ref{ch:3} will concentrate on the main results of this thesis. First, I will show that the Poisson integral extends the boundary value function in $L^p$ to the half space. The second part will extend this result to also include finite Borel measures.